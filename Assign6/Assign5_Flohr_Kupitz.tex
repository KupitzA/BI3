\documentclass[10pt,a4paper]{article}
\usepackage[utf8]{inputenc}

% \usepackage{ngerman}  % german documents
\usepackage{graphicx}  % import graphics einbinden
\usepackage{listings}  % support source code listing
\usepackage{amsmath}  % math stuff
\usepackage{amssymb} % 
\usepackage{a4wide} % wide pages
\usepackage{fancyhdr} % nice headers
\usepackage{tikz} %graphs
\usetikzlibrary{arrows}
\lstset{basicstyle=\footnotesize,language=Python,numbers=left, numberstyle=\tiny, stepnumber=5,firstnumber=0, numbersep=5pt} % set up listings
\pagestyle{fancy}             % header
\setlength{\parindent}{0pt}   % no indentation

\usepackage[pdfpagemode=None, colorlinks=true,  % url coloring
           linkcolor=blue, urlcolor=blue, citecolor=blue, plainpages=false, 
           pdfpagelabels,unicode]{hyperref}
           
% change enums style: first level (a), (b), (c)           
\renewcommand{\labelenumi}{(\alph{enumi})}
\renewcommand{\labelenumii}{(\arabic{enumii})}

%lecture name
\newcommand{\lecture}{
	Bioinformatics III
}           

%assignment iteration
\newcommand{\assignment}{
	Sixth Assignment
}

%set up names, matricle number, and email
\newcommand{\authors}{
  \begin{tabular}{rl}
    \href{mailto:s9alfloh@stud.uni-saarland.de}{Alexander Flohr} & (2549738)\\
    \href{mailto:s9ankupi@stud.uni-saarland.de}{Andrea Kupitz} & (2550260)
  \end{tabular}
}

% use to start a new exercise
\newcommand{\exercise}[1]
{
  \stepcounter{subsection}
  \subsection*{Exercise \thesubsection: #1}

}

\begin{document}
\title{\Large \lecture \\ \textbf{\normalsize \assignment}}
\author{\authors}

\setlength \headheight{25pt}
\fancyhead[R]{\begin{tabular}{r}\lecture \\ \assignment \end{tabular}}
\fancyhead[L]{\authors}


\setcounter{section}{5} % modify for later sheets, i.e. 2, 3, ...
%\section{Introduction to Python and some Network Properties} % optional, note that section invocation sets the section counter + 1, so adapt the setcounter command
\maketitle

\exercise{Boolean Networks}
\begin{enumerate}
\item Listing \ref{ex1-a1} shows the source code of our propagation matrix class, which behaves like such a matrix. Internally it uses a adjacency matrix to efficiently calculate next states. Therefore it depends on the class AdjacencyMatrix shown in Listing \ref{ex1-a2}. Further, the networks states are encoded in the class State, see \ref{ex1-a3}.
\lstinputlisting[tabsize=4, label=ex1-a1,caption={Listing of source code}] {PropagationMatrix.py}
\lstinputlisting[tabsize=4, label=ex1-a2,caption={Listing of source code}] {AdjacencyMatrix.py}
\lstinputlisting[tabsize=4, label=ex1-a3,caption={Listing of source code}] {State.py}
\newpage
\item Listing \ref{ex1-b} shows source code applying the the functiionality of the code shown in Listing \ref{ex1-a1}, which includes the network simulation.\\

1) It makes sense to stop the propagation when a state is observed a second time, from then on we will only observe orbiting behavior of the network states. The results of 2) are shown in this way. e.g. the first repeting state is the last shown.\\

2) Programs output for the required initial states:\\
\texttt{Initial state 1:\\
1 -> 3 -> 7 -> 23 -> 55 -> 63 -> 13 -> 1\\
\\
Initial state 4:\\
4 -> 18 -> 36 -> 26 -> 4\\
\\
Initial state 21:\\
21 -> 51 -> 47 -> 13 -> 1 -> 3 -> 7 -> 23 -> 55 -> 63 -> 13\\
\\
Initial state 33:\\
33 -> 11 -> 5 -> 19 -> 39 -> 31 -> 5\\
}

\lstinputlisting[tabsize=4, label=ex1-b,caption={Listing of source code}] {ex_6_1.py}

\item Output of the progam listing the orbits:\\
\texttt{Orbit 1 with length 1:\\
0\\
Set of basins:\\
0, 6, 8, 12, 16, 20, 22, 24, 28, 32, 34, 40, 42, 44, 46, 48, 50, 52, 54, 56, 58, 60, 62\\
Relative coverage: 35.9375\%\\
\\
Orbit 2 with length 7:\\
1, 3, 7, 23, 55, 63, 13\\
Set of basins:\\
1, 3, 7, 9, 13, 21, 23, 25, 29, 41, 43, 45, 47, 49, 51, 53, 55, 57, 59, 61, 63\\
Relative coverage: 32.8125\%\\
\\
Orbit 3 with length 4:\\
4, 18, 36, 26\\
Set of basins:\\
2, 4, 36, 38, 10, 14, 18, 26, 30\\
Relative coverage: 14.0625\%\\
\\
Orbit 4 with length 4:\\
5, 19, 39, 31\\
Set of basins:\\
33, 35, 5, 37, 39, 11, 15, 17, 19, 27, 31\\
Relative coverage: 17.1875\%}

\item Interpretation:\\
1) \textcolor{red}{???}\\
2) Two special are gene A and D. If A is active, the network becomes permanent \"impulses\", keeping the network in motion. Further, D hinders the network to remain fully active, i.g. if gene D is activated, it throws the network back into an earlier state, initiaing the orbiting behavior. This is further ensured since D inactivates its own activating gene.\\
Table \ref{ex1_tab} shows the two shorter orbits including the genes activation status at each state of the network. The upper described principles, are demonstrated for Orbit 4, where \"A activates the network step by step\" along the regulatory linkages until D gets activated. Afterwards, D inactivates most genes to reset the network to an earlier state.\\
For Orbit 3, the driving mechanisms are different. Here, B and C activate each other so that only one of both is active at the same time. When C is active, it further activates E what initiates D to inhibit B, E and F after 3 propagations. Since this inhibition occurs when C is active, the orbit closes and starts again. If D would have even distance (number of activating forward linkages), D would inhibit the active B what would turn the complete network inactive. In this case we would not observe orbiting behavior. 
\begin{table}
	\caption{Comparison of the two orbits with length 4}
	\label{ex1_tab}
	\begin{tabular}{|c|cccccc||c|cccccc|}
	\hline
	\multicolumn{7}{|c||}{Orbit 3} & \multicolumn{7}{c|}{Orbit 4}\\
	\hline
	Decimal & A & B & C & D & E & F & Decimal & A & B & C & D & E & F\\
	\hline
	4  &   &   & X &   &   &   &  5 & X &   & X &   &   &   \\
	18 &   & X &   &   & X &   & 19 & X & X &   &   & X &   \\
	36 &   &   & X &   &   & X & 39 & X & X & X &   &   & X\\
	26 &   & X &   & X & X &   & 31 & X & X & X & X & X &   \\
	\hline
	
	\end{tabular}
\end{table}
\end{enumerate}
\exercise{Differential Expression Analysis}
The task specific code is shown in Listing \ref{ex2-R}
\lstinputlisting[tabsize=4, label=ex2-R,caption={Listing of source code}] {ex_6_2.R}
We played with fdr parameter and found, that an increasing value for fdr increases the number of significant genes. Reasonable since a higher frd builds a less harsh threshold. This leads to more genes identified as significant. More permutations on the other hand do not affect the number of genes identified as significant. The numbers are fluctuating within a small range. But an increasing number of permutations refines the method so that the found results are more precise. So that the fals positive and negative rate reduces.\\
\newpage
10 up-regulated genes for sample 1:\\
\begin{tabular}{c | c}
	Gene Name & Fold Change\\
	\hline
	Polypyrimidine tract-binding protein 2                      & 21.679\\
	Cellular retinoic acid-binding protein 2                    & 6.432\\
	Creatine kinase U-type, mitochondrial                       & 14.359\\
	Disabled homolog 1                                          & 6.345\\ 
	Calcium-binding and coiled-coil domain-containing protein 2 & 4.017\\  
	Multidrug resistance protein 1                              & 5.337\\  
	Neurobeachin                                                & 4.166\\  
	Protein Tob2;Protein Tob1                                   & 3.483\\  
	Lanosterol 14-alpha demethylase                             & 2.898\\   
	Interferon-related developmental regulator 1                & 2.357\\
\end{tabular} \\

10 down-regulated genes for sample 1:\\
\begin{tabular}{c | c}
	Gene Name & Fold Change\\
	\hline	
	Cathepsin B & 0.091\\   
	Integrin beta-5 & 0.092\\    
	Integrin alpha-V & 0.097\\    
	Fibrillin-1 & 0.103\\    
	Polypeptide N-acetylgalactosaminyltransferase 1 & 0.083\\    
	Glia-derived nexin & 0.25\\     
	Solute carrier family 2, facilitated glucose transporter member 1 & 0.121\\    
	DnaJ homolog subfamily C member 3 & 0.173\\    
	Erlin-1 & 0.229\\    
	UPF0501 protein KIAA1430 & 0.164\\
\end{tabular}
\newpage
10 up-regulated genes for sample 2:\\
\begin{tabular}{c | c}
	Gene Name & Fold Change\\
	\hline
	Disabled homolog 2 & 3.036\\    
	Heme oxygenase 1 & 3.765\\
	Interferon-related developmental regulator 1 & 3.005\\    
	EPM2A-interacting protein 1 & 8.654\\    
	Disabled homolog 1 & 3.21\\
	Cellular retinoic acid-binding protein 2 & 2.605\\
	Asparagine synthetase [glutamine-hydrolyzing] & 2.355\\   
	EKC/KEOPS complex subunit LAGE3 & 3.012\\
	Dehydrogenase/reductase SDR family member 7 & 2.278\\    
	Exonuclease 3-5 domain-containing protein 2 & 1.928\\
\end{tabular}\\

10 down-regulated genes for sample 2:\\
\begin{tabular}{c | c}
	Gene Name & Fold Change\\
	\hline
	Receptor-type tyrosine-protein phosphatase eta & 0.303\\ 
	Alpha-galactosidase A & 0.245\\
	Protein disulfide-isomerase A5 & 0.149\\
	Integrin alpha-V;Integrin alpha-V heavy chain;Integrin alpha-V light chain & 0.144\\ 
	EGF-like repeat and discoidin I-like domain-containing protein 3 & 0.209\\ 
	Spectrin beta chain, non-erythrocytic 1 & 0.261\\
	DnaJ homolog subfamily C member 10 & 0.187\\
	Spectrin alpha chain, non-erythrocytic 1 & 0.368\\
	Cathepsin B;Cathepsin B light chain;Cathepsin B heavy chain & 0.282\\  
	Endoplasmic reticulum-Golgi intermediate compartment protein 1 & 0.243\\ 

\end{tabular}

\end{document}