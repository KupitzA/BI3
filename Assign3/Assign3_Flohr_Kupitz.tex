\documentclass[10pt,a4paper]{article}
\usepackage[utf8]{inputenc}

% \usepackage{ngerman}  % german documents
\usepackage{graphicx}  % import graphics einbinden
\usepackage{listings}  % support source code listing
\usepackage{amsmath}  % math stuff
\usepackage{amssymb} % 
\usepackage{a4wide} % wide pages
\usepackage{fancyhdr} % nice headers
\lstset{basicstyle=\footnotesize,language=Python,numbers=left, numberstyle=\tiny, stepnumber=5,firstnumber=0, numbersep=5pt} % set up listings
\pagestyle{fancy}             % header
\setlength{\parindent}{0pt}   % no indentation

\usepackage[pdfpagemode=None, colorlinks=true,  % url coloring
           linkcolor=blue, urlcolor=blue, citecolor=blue, plainpages=false, 
           pdfpagelabels,unicode]{hyperref}
           
% change enums style: first level (a), (b), (c)           
\renewcommand{\labelenumi}{(\alph{enumi})}
\renewcommand{\labelenumii}{(\arabic{enumii})}

%lecture name
\newcommand{\lecture}{
	Bioinformatics III
}           

%assignment iteration
\newcommand{\assignment}{
	Third Assignment
}

%set up names, matricle number, and email
\newcommand{\authors}{
  \begin{tabular}{rl}
    \href{mailto:s9alfloh@stud.uni-saarland.de}{Alexander Flohr} & (2549738)\\
    \href{mailto:s9ankupi@stud.uni-saarland.de}{Andrea Kupitz} & (2550260)
  \end{tabular}
}

% use to start a new exercise
\newcommand{\exercise}[1]
{
  \stepcounter{subsection}
  \subsection*{Exercise \thesubsection: #1}

}

\begin{document}
\title{\Large \lecture \\ \textbf{\normalsize \assignment}}
\author{\authors}

\setlength \headheight{25pt}
\fancyhead[R]{\begin{tabular}{r}\lecture \\ \assignment \end{tabular}}
\fancyhead[L]{\authors}


\setcounter{section}{3} % modify for later sheets, i.e. 2, 3, ...
%\section{Introduction to Python and some Network Properties} % optional, note that section invocation sets the section counter + 1, so adapt the setcounter command
\maketitle

\exercise{Naive Bayes classifier}
\begin{enumerate}
\item Because the features are independent: \begin{equation}P(S) = \prod_{i} P(S_i) \end{equation}
Following the Bayes theorem: \begin{equation} \frac{P(S|C) * P(C)} {P(S)} = P(C|S) \end{equation}
Further \begin{equation} log( \frac{P(C|S)} {P(\bar{C}|S)} ) = log( \prod_{i} \frac{ \frac{P(S_i|C) * P(C)}{P(S_i)} }{ \frac{P(S_i|\bar{C}) * P(\bar{C})}{P(S_i)} } ) = \sum_{i} log( \frac{P(S_i|C) * P(C)}{P(S_i|\bar{C}) * P(\bar{C})} ) \end{equation}

\item 

\item Listing \ref{ex1-1} shows source code.
\lstinputlisting[label=ex1-1,caption={Example Listing of source code}] {model.py}

\item

\item 
\end{enumerate}

\exercise{Network communities}
\begin{enumerate}
\item 

\item
\begin{enumerate}
\item
\item
\end{enumerate}
\end{enumerate}

\end{document}